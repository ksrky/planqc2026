\subsection{Operational semantics}
This section defines the operational semantics, describing how the reference-based semantics from Section~\ref{sec:main-idea} is formalized.
The state of a computation is captured by a \textit{configuration}: a pair of a global store and the term being evaluated.
Crucially, the store holds the physical quantum state, while the term operates only on locations ($\mathcal{L}$)---references to that store.
This separation allows contraction and weakening to be treated as logical operations on references, without directly affecting the physical quantum state.
Because a term is rewritten to contain these locations during execution, we must first define a distinct runtime language that extends the surface syntax with them.

\begin{dfn}[The runtime language $\Lambda_\mathcal{L}(\mathcal{U})$]
  Given a set of store locations $\mathcal{L}$, the runtime language is defined as $\Lambda_\mathcal{L}(\mathcal{U}) := \Lambda(\mathcal{U}\cup\mathcal{L})$.
  To simplify notation, $M, N, P$ will also range over runtime terms.
  We will explicitly refer to surface terms when the distinction is significant.
\end{dfn}

Only first-order, or \textit{basis values}, can be placed in the store and exist in a superposition.
Values and basis values are given by the following syntax.
\begin{equation*}
  \begin{array}{lrl}
    \text{Values}       & V, W \Coloneqq             & l \mid *\mid\lambda x. M\mid\langle V, W\rangle\mid\texttt{inj}_0\ V\mid\texttt{inj}_1\ V\mid\hat{U} \\
    \text{Basis values} & \hat{V}, \hat{W} \Coloneqq & *\mid\langle\hat{V}, \hat{W}\rangle\mid\texttt{inj}_0\ \hat{V}\mid\texttt{inj}_1\ \hat{V}
  \end{array}
\end{equation*}
The set of all basis values is denoted by $\mathcal{V}_0$.

A configuration $[\sigma, M]$ represents the state of a computation.
The quantum store $\sigma$ is a vector in a free Hilbert space over value stores (finite maps from locations $\mathcal{L}$ to basis values $\mathcal{V}_0$), while $M$ is the runtime term containing references to $\sigma$.
Each value store in the support of $\sigma$ must have the same domain; we can define $\text{dom}(\sigma)$ without ambiguity.
Write $\mathcal{C}$ for the set of all configurations.

The operational semantics are defined by a call-by-value, small-step reduction relation ($\longrightarrow$).
The key rules in Fig.~\ref{fig:reduction} directly correspond to the four postulates of quantum mechanics\cite{NIELSEN2010_QuantumComputationQuantum}.
\begin{description}
  \item[State preparation (\textsc{Q-Prep})] It prepares a quantum state by taking a classical basis value $\hat{V}$ and placing it into the store at a new location $l$.
  \item[Unitary Evolution (\textsc{Q-Evolve})] Applying a unitary $\hat{U}$ to a reference $l_1$ transforms the entire quantum store $\sigma$ according to the operator's interpretation.
        The location $l_1$ stays in the store, which reflects the 'Contraction as sharing' mechanism as illustrated in Fig.~\ref{fig:grids}.
  \item[Measurement (\textsc{Q-Meas})] The match expression performs a measurement on a referenced state, causing the computation to branch probabilistically and collapsing the quantum store.
  \item[Composite system (\textsc{Q-Destr})] It destructs a pair stored at $l_1$ and creates new, distinct references $l_2$, $l_3$ to its components.
        It copies values at each value store, not meaning that the superpositions themselves are cloned.
\end{description}

This semantics imposes a meta-level constraint on $\hat{U}$.
Specifically, the \textit{interpretation} $\llbracket\hat{U}\rrbracket_\mathcal{U} : \mathcal{V}_0\rightharpoonup \ell^2(\mathcal{V}_0)$ must satisfy the following condition: if $U = \llbracket\hat{U}\rrbracket_\mathcal{U}$, then $(U|_{\mathrm{dom}(U)})^\sharp$ is unitary.
\begin{dfn}[Small-step reduction] \label{def:single-step}
  Given a set of partial functions $\llbracket\hat{U}\rrbracket_\mathcal{U}$ for $\hat{U}\in\mathcal{U}$, the reduction relation $\longrightarrow \subseteq \mathcal{C}\times \mathcal{D}(\mathcal{C})$ is the smallest relation closed under the classical rules (Appendix~\ref{sec:opsem-full}) and the quantum rules (Fig.~\ref{fig:reduction}).
  The relation is then extended to all evaluation contexts $E$ via the standard congruence rule: if $[\sigma,M] \longrightarrow \mu$, then $[\sigma,E[M]] \longrightarrow \sum_{[\sigma',M']\in\mathrm{supp}(\mu)}\mu([\sigma',M'])\delta_{[\sigma',E[M']]}$.
  The definition of evaluation contexts is given in Appendix~\ref{sec:opsem-full}.
\end{dfn}

\begin{figure}[t]
  \centering
  \begin{mathpar}
    \inferrule*[right=Q-Destr]
    {f(\rho) = \rho\cup\{(l_2,\hat{V}),(l_3, \hat{W})\}\text{ where } \rho(l_1) = \langle \hat{V}, \hat{W}\rangle \\ l_2, l_3 \notin \mathrm{dom}(\sigma)}
    {[\sigma, \texttt{let}\ \langle x, y\rangle=l_1\ \texttt{in}\ M] \longrightarrow \delta_{[\ell^2(f)(\sigma), M[l_2/x,l_3/y]]}}
    \and
    \inferrule*[right=Q-Meas]
    {|\alpha_0|^2 + |\alpha_1|^2 = 1 \\ ||\sigma_0|| = ||\sigma_1|| \\ f_0(\rho) = \rho\cup\{(l_2,\hat{V})\}\text{ where } \rho(l_1) = \texttt{inj}_0\ \hat{V} \\ f_1(\rho) = \rho\cup\{(l_3,\hat{W})\}\text{ where } \rho(l_1) = \texttt{inj}_1\ \hat{W} \\ l_2 \notin \mathrm{dom}(\sigma_0) \\ l_3 \notin \mathrm{dom}(\sigma_1)}
    {[\alpha_0\sigma_0 + \alpha_1\sigma_1, \texttt{match}\ l_1\ [y\Rightarrow M\mid z\Rightarrow N]] \\
    \longrightarrow |\alpha_0|^2 \delta_{[\ell^2(f_0)(\sigma_0), M[l_2/y]]} + |\alpha_1|^2\delta_{[\ell^2(f_1)(\sigma_1),N[l_3/z]]}}
    \and
    \inferrule*[right=Q-Prep]
    {f(\rho) = \rho\cup\{(l, \hat{V})\} \\\\ l \notin \mathrm{dom}(\sigma)}
    {[\sigma, \hat{U}\ \hat{V}] \longrightarrow \delta_{[\ell^2(f)(\sigma), \hat{U}\ l]}}
    \and
    \inferrule*[right=Q-Evolve]
    { f(\rho) = \sum\nolimits_{\hat{V}}\llbracket\hat{U}\rrbracket_\mathcal{U}(\rho(l_1))(\hat{V})\delta_{\rho\cup\{(l_2, \hat{V})\}} \\\\ l_2 \notin \mathrm{dom}(\sigma)}
    {[\sigma, \hat{U}\ l_1] \longrightarrow \delta_{[f^\sharp(\sigma),l_2]}}
  \end{mathpar}
  \caption{Quantum reduction rules}
  \Description{Four inference rules defining the quantum reduction steps. The first rule (\textsc{Q-Destr}) describes the decomposition of a pair in the store into two new locations. The second rule (\textsc{Q-Meas}) details the measurement process, branching based on the superposition of states. The third rule (\textsc{Q-Prep}) covers state preparation by placing a basis value into the store. The fourth rule (\textsc{Q-Evolve}) explains the application of a unitary operator to a location, transforming the quantum state accordingly.}
  \label{fig:reduction}
\end{figure}

A key property to the physical realizability of the semantics is norm preservation, which guarantees that the norm of the quantum state does not change during reduction.
\begin{restatable}[Norm preservation]{thm}{NormpresTheorem} \label{thm:norm-pres}
  If $[\sigma,M] \longrightarrow \mu$, then for all $[\sigma',M']\in\mathrm{supp}(\mu)$, $||\sigma'|| = ||\sigma||$.
\end{restatable}

Since a single reduction step can yield a distribution of configurations, the \textit{multi-step reduction} ($\longrightarrow^*$) is not a simple reflexive and transitive closure. Instead, it is defined inductively to compose these probabilistic outcomes, akin to a monadic bind.
\begin{dfn}[Multi-step reduction]
  The multi-step reduction relation $\longrightarrow^*\subseteq \mathcal{C}\times \mathcal{D}(\mathcal{C})$ is the smallest relation satisfying:
  \begin{itemize}
    \item $[\sigma,M] \longrightarrow^*\delta_{[\sigma,M]}$ for all $[\sigma,M]\in\mathcal{C}$.
    \item If $[\sigma,M] \longrightarrow \mu$ and for all $[\sigma',M']\in\mathrm{supp}(\mu)$, $[\sigma',M'] \longrightarrow^*\nu_{\sigma',M'}$, then $[\sigma,M] \longrightarrow\sum_{[\sigma',M']\in\mathrm{supp}(\mu)} \mu([\sigma',M'])\nu_{\sigma',M'}$.
          % Each reduction step has finite probablistic branches; thus the summation is well-defined.
  \end{itemize}
\end{dfn}

Then, we define an equivalence-like relation based on multi-step reduction.
\begin{dfn}
  For all $M, N \in \Lambda$, define $M \longleftrightarrow^* N$ iff
  for any $\sigma$, there exist $\mu$ such that $[\sigma, M] \longrightarrow^* \mu$ and $[\sigma, N] \longrightarrow^* \mu$.
\end{dfn}

%\begin{dfn}[Termination]
%  A surface term $M$ is said to \emph{terminate}, denoted $M\Downarrow$, if there exists $\mu : \mathcal{V} \to [0,1]$ such that $[\delta_\emptyset, M] \longrightarrow\mu$, and for every $[\sigma, M'] \in \mathrm{supp}(\mu)$, $M'$ is a closed value.
%\end{dfn}
