\section{Discussion}

\begin{itemize}
\item Prove the conjecture (cite Dal Lago's work); discuss soundness of $\eta_V$-equivalence.
\item Discuss that the present semantics consumes too many qubits.
  This is OK as the proposed calculus is a semantic basis.  For
  implementation, we need some kind of garbage collection (?).
\end{itemize}

This paper established that classical $\beta$-equivalence is preserved by contextual equivalence ($\leftrightarrow_\beta \Rightarrow \approx$), which justifies applying classical optimizations to the classical fragment.
However, the converse, and thus a full conservative extension result for observational equivalence, fails.
This is a necessary design choice, as the observational effect of measurement breaks $\eta$-equivalence for constructs like match, i.e., $\texttt{match}\ x\ [y\Rightarrow \texttt{inj}_0\ y\mid z\Rightarrow \texttt{inj}_1\ z] \not\approx x$.
The true conservative extension property should relate the equational theories of the two calculi.
Formulating this remains a significant open problem, as our probabilistic reduction relation ($[\sigma, M] \longrightarrow \mu$) requires a novel definition of an equivalence closure that can operate on distributions, a topic we leave for future investigation.


%We aim to develop both practical compilation techniques and theoretical extensions.
%An intermediate language will be designed to explicitly describe interactions between classical and quantum systems through lower-level quantum operations.
%Although the simply typed system is presented in Appendix~\ref{sec:type-system}, it can be extended with polymorphism and recursive types.
%We are also working on a denotational semantics to formalize the language's mathematical interpretations.
