\section{Discussion}
This paper establishes a non-linear calculus that reinterprets structural rules as reference operations, preserving key classical identities like $\pi_{1}\circ\delta=\lambda x.x$.
The central conjecture, crucial for justifying classical reasoning, remains that full $\beta_V$-equivalence is preserved by quantum contextual equivalence.
A proof may require adapting bisimulation techniques, such as those in \cite{DALLAGO2014_CoinductiveEquivalencesHigherorder,SANGIORGI2019_EnvironmentalBisimulationsProbabilistic}, to our quantum semantics.
It should be noted that $\eta_V$-equivalence for $\texttt{match}$ does not hold in general because measurement induces probabilistic branching, e.g., $\texttt{match}\ x\ [y\Rightarrow \texttt{inj}_0\ y\mid z\Rightarrow \texttt{inj}_1\ z] \not\approx x$.

A second point concerns resource management.
Our high-level semantics, where new store locations imply new qubits, is a deliberate abstraction for a clear operational model.
For a practical implementation, we should introduce a compiler layer responsible for explicit resource management, such as garbage collection or qubit reuse.
