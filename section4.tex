\subsection{Conservativity}
As mentioned in Section \ref{sec:intro}, the central design principle of our language is \textit{conservativity}; a formal guarantee that the quantum extension does not alter the original behavior of the classical fragment.
Informally, this means that for any term, one may apply the classical reduction rules to the subterms that do not contain $\hat{U}$ without changing the overall behavior of the term.

Conservativity is then stated as the following theorem.
The proof is given in Appendix \ref{sec:conserv-proof}.
\begin{restatable}[Conservativity]{thm}{ConservTheorem} \label{thm:conservativity}
  For all $M, N \in \Lambda$, $M \longleftrightarrow^*_\Lambda N$ iff $M \longleftrightarrow^* N$.
\end{restatable}
This property cannot be proved for general operational equivalence other than $\beta$-equivalence since our language yields side-effects through measurement.
Actually, $\eta$-equivalence of \texttt{match} does not hold in general.
Consider the term $\texttt{match}\ x\ [y\Rightarrow \texttt{inj}_0\ y\mid z\Rightarrow \texttt{inj}_1\ z]$.
In classical reduction, it is $\eta$-equivalent to $x$.
However, $x$ can store a quantum state, and thus measuring it results in a probabilistic outcome.
