\section{Conservativity}
As mentioned in Section \ref{sec:intro}, the central design principle of our language is \textit{conservativity}; a formal guarantee that the quantum extension does not alter the original behavior of the classical fragment.
Informally, this means that for any term, one may apply the classical reduction rules to the subterms that do not contain $\hat{U}$ without changing the overall behavior of the term.

To formalize this, we first define contextual equivalence, which states that two terms are equivalent if their observable behaviors are identical in any context.
A context $C[\cdot]$ is defined as a term with a single hole $[\cdot]$ in it.

\begin{dfn}[Contextual equivalence] \label{def:contextual-equiv}
  For $M, N \in \Lambda$, define two relations $\approx$ and $\approx_\beta$ as follows:
  \begin{itemize}
    \item $M\approx N \Leftrightarrow_\mathrm{def} \forall C[\cdot]\in\Lambda(\mathcal{U}).\ C[M] \Downarrow \Leftrightarrow C[N] \Downarrow$.
    \item $M\approx_\beta N \Leftrightarrow_\mathrm{def} \forall C[\cdot]\in\Lambda.\ C[M] \Downarrow_\beta \Leftrightarrow C[N] \Downarrow_\beta$, where $\Downarrow_\beta$ is defined similarly to $\Downarrow$ but using only the congruence rules and the classical rules presented in Appendix \ref{sec:opsem-full}.
  \end{itemize}
\end{dfn}

Conservativity is then stated as the following theorem.
\begin{restatable}[Conservativity]{thm}{ConservTheorem} \label{thm:conservativity}
  For all $M, N \in \Lambda$, $M \approx_\beta N$ iff $M \approx N$.
\end{restatable}
