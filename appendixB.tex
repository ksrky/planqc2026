\section{Proof of conservativity} \label{sec:conserv-proof}
\begin{lem} \label{lem:classical-step}
  For any $M \in \Lambda$, if $M \longrightarrow M'$, then $M \longrightarrow_\beta M'$ and $M' \in \Lambda$.
\end{lem}
\begin{proof}
  By induction on the derivation of $M \longrightarrow M'$.
\end{proof}

\begin{lem} \label{lem:classical-eval}
  For any $M \in \Lambda$, $M \Downarrow_\beta$ iff $M \Downarrow$.
\end{lem}
\begin{proof}
  (\textbf{Sufficiency})
  As the classical reduction relation $\longrightarrow_\beta$ is a subset of the full reduction relation $\longrightarrow$, also $\longrightarrow_\beta^* \subseteq \longrightarrow^*$ holds.

  (\textbf{Necessity})
  By Lemma \ref{lem:classical-step}, every step in the reduction sequence from $M$ to a value $V$ under $\longrightarrow$ corresponds to a step under $\longrightarrow_\beta$.
  Thus, if $M \Downarrow$, there exists a sequence of reductions under $\longrightarrow_\beta$ from $M$ to $V$, establishing $M \Downarrow_\beta$.
\end{proof}

\ConservTheorem*
\begin{proof}[Proof]
  (\textbf{Sufficiency})
  Assume $M\approx_\beta N$ and let $C[\cdot] \in \Lambda(\mathcal{U})$ be any context.
  We prove if $C[M] \Downarrow$, then $C[N] \Downarrow$ holds.
  The converse direction is ommitted since it can be shown in the same way.
  We prove by induction on the derivation of the multi-step reduction relation.
  \begin{itemize}
    \item If $C[M] \Downarrow$ is $[\delta_\emptyset, C[M]] \longrightarrow^* \delta_{[\delta_\emptyset,V]}$ for some $V$, then $C$ is a hole and $M = V$.
          By the assumption $M \approx_\beta N$, $N$ is also $V$; thus, $C[N] \Downarrow$ holds.
    \item If the first reduction step of $C[M] \Downarrow$ is $[\delta_\emptyset, C[M]] \longrightarrow \mu$, then we do case analaysis on the rule applied in this step.
          %\begin{itemize}
          %  \item Case $\beta$-\textsc{Lam}: If $C[M]$ is $C'[M]\ \dot{V}$, then by the induction hypothesis, $C'[N]$ also terminates.
          %        If $C[M]$ is $(\lambda x.\dot{M}')\ \dot{V}$, then $C$ is a hole and $M = \dot{V}$.
          %\end{itemize}
  \end{itemize}
  (\textbf{Necessity})
  Proof by contrapositive.
  We assume $M \not\approx_{\beta} N$ for $M, N \in \Lambda$, and show that this implies $M \not\approx N$.
  The assumption $M \not\approx_{\beta} N$ means there exists a classical context $C[\cdot] \in \Lambda$ that distinguishes $M$ and $N$ in the classical fragment.
  Without loss of generality, let's assume $C[M] \Downarrow_\beta$ while $C[N] \Downarrow_\beta$ does not hold.

  Since $C[M] \Downarrow_\beta$, Lemma~\ref{lem:classical-eval} implies that $C[M] \Downarrow$.
  Since the evaluation of $C[N]$ under the classical rules diverges, the lemma implies that the evaluation of $C[N]$ in the full semantics also diverge.

  Thus, the very same classical context $C[\cdot] \in \Lambda$, which is also a valid context in $\Lambda(\mathcal{U})$, serves to distinguish $M$ and $N$ in the full semantics.
  This establishes that $M \not\approx N$, completing the proof of the contrapositive.
\end{proof}