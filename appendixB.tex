\section{Proof of conservativity}
\begin{lem} \label{lem:classical-step}
  For any $M \in \Lambda$, if $M \longrightarrow M'$, then $M \longrightarrow_\beta M'$ and $M' \in \Lambda$.
\end{lem}
\begin{proof}
  By induction on the derivation of $M \longrightarrow M'$.
\end{proof}

\begin{lem} \label{lem:classical-eval}
  For any $M \in \Lambda$, $M \Downarrow$ iff $M \Downarrow_\beta$.
\end{lem}
\begin{proof}
  (\textbf{Sufficiency})
  As the classical reduction relation $\longrightarrow_\beta$ is a subset of the full reduction relation $\longrightarrow$, also $\longrightarrow_\beta^* \subseteq \longrightarrow^*$ holds.

  (\textbf{Necessity})
  By Lemma \ref{lem:classical-step}, every step in the reduction sequence from $M$ to a value $V$ under $\longrightarrow$ corresponds to a step under $\longrightarrow_\beta$.
  Thus, if $M \Downarrow$, there exists a sequence of reductions under $\longrightarrow_\beta$ from $M$ to $V$, establishing $M \Downarrow_\beta$.
\end{proof}

\ConservTheorem*
\begin{proof}[Proof]
  (\textbf{Sufficiency})
  \\
  (\textbf{Necessity})
  Proof by contradiction.
  We assume $M \not\approx_{\beta} N$ for $M, N \in \Lambda$, and show that this implies $M \not\approx N$.
  The assumption $M \not\approx_{\beta} N$ means there exists a classical context $C[\cdot] \in \Lambda$ that distinguishes $M$ and $N$ in the classical fragment.
  Without loss of generality, let's assume $C[M] \Downarrow_\beta$ while $C[N] \Downarrow_\beta$ does not hold.

  Since $C[M] \Downarrow_\beta$, Lemma~\ref{lem:classical-eval} implies that $C[M] \Downarrow$.
  Since the evaluation of $C[N]$ under the classical rules diverges, the lemma implies that the evaluation of $C[N]$ in the full semantics must also diverge.

  Thus, the very same classical context $C[\cdot] \in \Lambda$ (which is also a valid context in $\Lambda(\mathcal{U})$) serves to distinguish $M$ and $N$ in the full quantum semantics.
  This establishes that $M \not\approx N$, completing the proof of the contrapositive.
\end{proof}