\section{Introduction} \label{sec:intro}
The theoretical foundations of quantum programming languages have largely been built upon linear logic and its variants\cite{VANTONDER2004_LambdaCalculusQuantum,SELINGER2009_QuantumLambdaCalculus,ALTENKIRCH2005_FunctionalQuantumProgramming,SABRY2018_SymmetricPatternMatchingQuantum,ROSS2017_AlgebraicLogicalMethods}, reflecting the \textit{No-Cloning} Theorem\cite{WOOTTERS1982_SingleQuantumCannota} and the \textit{No-Deleting} Theorem\cite{KUMARPATI2000_ImpossibilityDeletingUnknowna} of quantum mechanics.
Within this paradigm, variables representing quantum states are treated as linear resources that can be neither duplicated (contracted) nor discarded (weakened).
While this linearity ensures the physical correctness, it creates a significant gap with the intutions of classical, non-linear programming.

This paper proposes a quantum language designed as a direct quantum extension of the normal (non-linear) lambda calculus.
Our central design principle is to ensure this extension is conservative; specifically, $\beta$-equivalence of the original non-linear lambda calculus is preserved by the quantum extension.
This approach ensures any embedded classical program executes with its standard semantics, making the language's classical fragment indistinguishable from the original lambda calculus.
A key consequence of enforcing this conservativity property is that the quantum features of the language are cleanly isolated from the classical host.
This quantum-native fragment, consisting of encapsulated unitary operations, forms a distinct module for expressing quantum control.
This separation provides a natural framework that enables coexistence of the historically separate paradigms of `\textit{Quantum Data, Classical Control}'\cite{SELINGER2004_QuantumProgrammingLanguage} and `\textit{Quantum Data, Quantum Control}'\cite{DÍAZ-CARO2022_QuickOverviewQuantum}.
That means a programmer can compose the main logic in a higher-order classical style, while delegating complex quantum subroutines to specialized, domain-specific modules.
The resulting language provides a familiar framework for classical programmers and, at the same time, introduces a structured and powerful abstraction for high-level quantum-classical programming.
