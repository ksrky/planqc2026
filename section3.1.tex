\section{\texorpdfstring{The $\Lambda(\mathcal{U})$ calculus}{The Lambda(U) calculus}}
\subsection{Syntax}
As outlined in the introduction, our language is designed as a conservative extension of a non-linear lambda calculus denoted by $\Lambda$.
$\Lambda$ includes lambda terms, the unit and introductions and eliminations of products and sums.
The key to our design is the isolation of quantum features from $\Lambda$, which is provided by a set $\mathcal{U}$ of external unitary operations.
\begin{dfn}[The surface language $\Lambda(\mathcal{U})$]
  Given a set of terms $\mathcal{U}$, the language $\Lambda(\mathcal{U})$ is the set of terms defined by the following syntax:
  \begin{equation*}
    \begin{array}{rl}
      M, N, P \Coloneqq & *\mid x\mid\lambda x. M\mid M\ N\mid\langle M, N\rangle\mid\texttt{inj}_0\ M\mid\texttt{inj}_1\ M \mid \\& \texttt{let}\ \langle x, y\rangle=M\ \texttt{in}\ N\mid\texttt{match}\ M\ [x\Rightarrow N\mid y\Rightarrow P]\mid\hat{U}
    \end{array}
  \end{equation*}
  where $\hat{U}\in\mathcal{U}$. Alternatively, $\Lambda(\mathcal{U})$ is defined as $\Lambda\cup\mathcal{U}$.
\end{dfn}

From the perspective of the host language $\Lambda(\cdot)$, $\hat{U}$ is treated as an opaque, first-class value.
The language $\mathcal{U}$ can represent any unitary transformation, ranging from a simple unitary gate to a complex subroutine that could be described by a domain-specific language, such as one based on the `Quantum Data, Quantum Control' paradigm.
\begin{ex} \label{ex:classical}
  The classical bit can be represented as $i \equiv \texttt{inj}_i\ *$ where $i=0,1$.
  A standard conditional expression $\texttt{if}\ M\ \texttt{then}\ N\ \texttt{else}\ P$ can be written as a syntactic sugar for $\texttt{match}\ M\ [x\Rightarrow N\mid y\Rightarrow P]$, where $x$ and $y$ are fresh variables.
  Using these notations, the logical NOT is $\textit{not} \equiv \lambda x.\texttt{if}\ x\ \texttt{then}\ 1\ \texttt{else}\ 0$, and the logical AND is $\textit{and} \equiv \lambda x.\lambda y.\texttt{if}\ x\ \texttt{then}\ y\ \texttt{else}\ 0$.
\end{ex}
The quantum aspect of the language is captured by the $\hat{U}$ term.
Match expressions need special attention in this context.
It behaves as usual only when the outermost of the scrutinee is evaluated to inl or inr.
When it is evaluated to superposition of inl and inr, it performs a measurement to project one of them before that.
\begin{ex}[Quantum teleportation] \label{ex:teleportation}
  Let $\mathcal{U}$ be $\{H,CNOT,Z,X\}$ as the standard quantum gate set.
  The following program implements the quantum teleportation circuit on the right:
  \begin{center}
    \begin{minipage}{0.40\textwidth}
      \begin{equation*}
        \begin{array}{l}
          \texttt{let}\ \langle a, b\rangle=CNOT\ \langle H\ 0, 0\rangle\ \texttt{in}           \\
          \texttt{let}\ \langle x, y\rangle=CNOT\ \langle q, a\rangle\ \texttt{in}              \\
          \texttt{let}\ b' = \texttt{if}\ y\ \texttt{then}\ X\ b\ \texttt{else}\ b\ \texttt{in} \\
          \texttt{if}\ H\ x\ \texttt{then}\ Z\ b'\ \texttt{else}\ b'
        \end{array}
      \end{equation*}
    \end{minipage}
    \begin{minipage}{0.45\textwidth}
      \centering
      \hspace{1em}
      \Qcircuit @C=0.8em @R=0.8em {
      \lstick{q}       & \qw      & \qw      &           \qw & \ctrl{1} & \ustick{x}\qw &\gate{H}         &            \qw & \meter           & \\
      \lstick{\ket{0}} & \gate{H} & \ctrl{1} & \ustick{a}\qw & \targ    & \ustick{y}\qw &\meter           &                & \cwx             & \\
      \lstick{\ket{0}} & \qw      & \targ{}  & \ustick{b}\qw & \qw      &           \qw &\gate{X}\cwx{-1} & \ustick{b'}\qw & \gate{Z}\cwx{-2} & \qw
      }
    \end{minipage}
  \end{center}
  where $\texttt{let}\ x = M\ \texttt{in}\ N$ is a syntactic sugar for $(\lambda x. N)\ M$.
\end{ex}
It should be noted that the \textit{not} function in Example \ref{ex:classical} does not behave as the $X$ gate (or the quantum NOT gate) when applied to a qubit, because the input is measured in the condition of the if expression, which collapses any superposition.

As a powerful example of this encapsulation of $\mathcal{U}$, a syntax inspired by pattern isomorphism\cite{SABRY2018_SymmetricPatternMatchingQuantum} allows for the direct declaration of unitary operations as follows.
Let $\mathcal{U}$ be defined using the pattern isomorphism syntax.
For instance, an $X$ gate can be defined as $X \equiv \{0 \leftrightarrow 1\mid 1 \leftrightarrow 0\}$.